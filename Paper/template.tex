\documentclass{article}


\usepackage{arxiv}

\usepackage[utf8]{inputenc} % allow utf-8 input
\usepackage[T1]{fontenc}    % use 8-bit T1 fonts
\usepackage{hyperref}       % hyperlinks
\usepackage{url}            % simple URL typesetting
\usepackage{booktabs}       % professional-quality tables
\usepackage{amsfonts}       % blackboard math symbols
\usepackage{nicefrac}       % compact symbols for 1/2, etc.
\usepackage{microtype}      % microtypography
\usepackage{lipsum}
\usepackage{amssymb}
\usepackage[dvipsnames]{xcolor}
\usepackage{listings,lstlangcoq}
\usepackage{upgreek }

\newtheorem{definition}{Definition}
\usepackage[framemethod=TikZ]{mdframed}



\title{Certified Reasoning About Real-Time Systems using Scheduling Algebra}


\author{
  David Pereira\thanks{all who helped.} \\
  CISTER / ISEP - P.Porto\\
  xxx\\
  Porto, Portugal \\
  \texttt{drp@isep.ipp.pt} \\
  %% examples of more authors
   \And
 Who else?\\
  xxx\
  xxx\\
  xxx \\
  \texttt{xx@yy.pt} \\
  %% \AND
  %% Coauthor \\
  %% Affiliation \\
  %% Address \\
  %% \texttt{email} \\
  %% \And
  %% Coauthor \\
  %% Affiliation \\
  %% Address \\
  %% \texttt{email} \\
  %% \And
  %% Coauthor \\
  %% Affiliation \\
  %% Address \\
  %% \texttt{email} \\
}

\begin{document}
\maketitle

\begin{abstract}
Software application are becoming more complex, and their particular usage to control Cyber-Physical Systems that support several tasks of our daily activities, demand that these applications have to satisfy strict scheduling rules to ensure their correct operation, under strict timing constraints. Paradigmatic examples are ADAS applications that are necessary for the development of future autonomous vehicles, advanced robotics and its application in the context of Industry 4.0, or even applications for the financial sector. However, developing applications with strict timing constraints is know to be a difficult task. Therefore, frameworks that allow to specify and reasoning about these constraints can be of great value to the designers and developers of such applications.  In this paper, we revisit {\em Scheduling algebra} and mechanise it in the type theory of Coq. The outcome of this mechanisation effort is a certified framework that allows to encode concurrent applications with timing constraints and reason about their timed, interleaved computations.   
\end{abstract}


% keywords can be removed
\keywords{First keyword \and Second keyword \and More}


\section{Introduction}

Applications are growing in complexity due to the increasing on the number of internal components, increased interaction with the physical environment (including with humans), and are highly connected. Many operations need to ensure strict timing requirements, e.g., autonomous driving, drone control, factory automation with robotics, among many others. It therefore becomes imperative for software developers to have access at rigorous methods that allow to precisely specify timed schedules and verify if such schedules are ensured at the run-time of the application. 

Current industrial practice in


\section{Scheduling Algebra}
\label{sec:1}

Scheduling algebra (SA) was introduced by van Glabbeek and Rittgen \cite{} as an algebraic theory of scheduling. Its syntax allows to build terms that specify tasks built from actions that have an associated duration, while the semantic interpretation if that of sets of possible {\em schedules} satisfied by such specifications. 


In this section we introduce and syntax and semantics of scheduling algebra, and show how we have encoding them in Coq. 

\textcolor{blue}{*** \\
Add an example of what type of property we are intending to formalize, and how one intends to reason about those. Explain at the high-level what is the semantic model we are going to consider (what type of traces are we considering), i.e., how we intend to formalize schedules so that we can algebraically reason about them.\\ 
***}

\subsection{Syntax}
\label{sec:1_1}
The language of {\sf SA}, denoted $\mathcal{L}_{SA}$, is that of terms that allow to specify the concurrent execution of actions that have an associated execution time. It assumes a finite set of actions $\mathcal{A} = \{a_1,\ldots,a_n\}$ with $n \geq 1$ and a representation of time as the set of real numbers. A special empty action $\epsilon$ is assumed. Pairs $(a,t) \in \mathcal{A}\times\mathcal{T}$, denoted $[a]^t$, specifies actions $a$ that take $t$ units of time to execute (also referred to $a$ has having duration $t$). In this work, we consider discrete, tick-based time and as so, time is interpreted as the set of natural numbers.

Terms $\tau \in \mathcal{L}_{SA}$ are inductively defined as follows: 

$$
\tau ::= \varepsilon\:|\:a^t\:|\: a^t \odot \tau\:|\:t \odot \tau\:|\:\Delta(\tau)\:|\:[\tau]\:|\:\tau \cup \tau\:|\:\tau \pitchfork \tau,
$$

such that $\epsilon$ denotes the empty action, $[a]^t \odot \tau_1$ denotes the sequential execution of $a$ during $t$ time units followed by the term $\tau_1$, $\tau_1 \cup \tau_2$ denotes the non-deterministic choice between executing $\tau_1$ or $\tau_2$, and $\tau_1 \pitchfork \tau_2$ denotes the concurrent execution of $\tau_1$ and $\tau_2$.



The encoding in Coq of the above concepts is straight forward, by defining two ADTs, \verb!Action! and \verb!term!, as presented below. 
\begin{lstlisting}[language={Coq}]
Inductive Action : Type :=
(** Empty action *)
| noAct : Action
(** Some action *)
| aAct  : $\mathbb{N}$ -> Action.

Inductive term : Type :=
(** Empty process (lifting of empty action to processes)  *)
| empProc    : term
(** Execution of a single action during some time *)
| singleProc : Action -> nat -> term
(** Execution of an action during some time, followed by other process(es) *)
| seqProc    : Action -> nat  -> term -> term
(** Offsetting the execution of a process for a given number of time units *)
| offsetProc : nat -> term -> term
(** Unbounded delay of a process *)
| delayProc  : term -> term
(** Elimination of delays *)
| elimProc   : term -> term
(** Non-deterministic choice between two processes *)
| choiceProc : term -> term -> term
(** Concurrent execution of processes *)
| forkProc   : term -> term -> term.
\end{lstlisting}

We consider syntactic shortcuts for representing: 1) the execution of a single action $\langle a \rangle^t$ which denotes the term $[a]^t \odot \epsilon$, and starting the execution of any action with an offset of $t$ time units $t \odot \tau$ which denotes the term $[\epsilon]^t \odot \tau$.

\subsection{Semantics}

The core concepts supporting the semantics of {\sf SA} is that of a schedule, which is represented by a triple $\langle \upsigma_1, \upsigma_2, \upsigma_3 \rangle$, such that $\upsigma_1$ is a multiset of triples $(a,i,f)$ with $a \in \mathcal{A}$ being an action, $i \in \mathbb{N}$ being the initial point in time when $a$ starts to execute, and $f \in \mathbb{N} \setminus \{0\}$ being the point in time when $a$ stops its execution, $\upsigma_2 \subseteq{N}$ is a set of points referring to starting points of the actions in $\upsigma_1$, and $\upsigma_3$ is a set of so-called {\em anchor points}, that is, points where actions have stopped execution. Note that since $\upsigma_1$ is a multi-set, the tiple $(a,i,f)$ can occur several times, denoting the parallel execution of $a$ starting at $i$ and finishing at $f$. We refer to elements $(a,i,f)$ as {\em bounded-timed actions} ({\sc bta}) since they denote the execution of the action $a$ during $f$ units of time, starting at point in time $i$.

Formally, {\sf SA} is interpreted over set whose elements have the type $\mathcal{S} = M(\mathcal{A},\mathbb{N},\mathbb{N}-{0}) \times 2^\mathbb{N} \times 2^{\mathbb{N}-\{0\}}$ such that each element $s \in \mathcal{S}$ is a bounded-timed action, and the two remaining components are subsets of the natural numbers containing all the initial execution times and final-execution times, so called-anchors, of those actions.

Before introducing the complete interpretation of schedule terms, we provide an intuitive example of what is an expected schedule for the simple program $a^1\;\odot\;b^2 \pitchfork a^1 $ which executes two concurrent tasks, one composed of a sequence of {\sc bta} and another with just a single one. The execution of the program goes as follows. On program start,  the action $a$ runs for a single unit of time, representing the {\sc bta} $(a,0,1)$. The  corresponding schedule is $\langle \{(a,1,0),(a,1,0)\}, \{0,1\} , \{1\} \rangle$. Now, $b$ must execute for $2$ units of time, starting in time $1$ and hence we have $(b,1,2)$. Consequently, we obtain the schedule $\langle \{(a,1,0),(a,1,0),(b,2,1)\}, \{0,1,3\} , \{1,3\} \rangle$.

\begin{definition}
	Let $S_1 = \langle \sigma_1^1 ; \sigma_2^1 ; \sigma_3^1 \rangle$ and $S_2 = \langle \sigma_1^2 ; \sigma_2^2 ; \sigma_3^2 \rangle$ be two schedules. The union of $S_1$ with $S_2$, denoted $S_1 \cup S_2$ is the schedule $\langle \sigma_1^1 \cup_m \sigma_1^2 ; \sigma_2^1 \cup \sigma_2^2 ; \sigma_3^1 \cup \sigma_3^2 \rangle$, such that union of schedules $\cup_m$ is the classic union over multisets.
\end{definition}





\section{Examples of citations, figures, tables, references}
\label{sec:others}
\lipsum[8] \cite{kour2014real,kour2014fast} and see \cite{hadash2018estimate}.

The documentation for \verb+natbib+ may be found at
\begin{center}
  \url{http://mirrors.ctan.org/macros/latex/contrib/natbib/natnotes.pdf}
\end{center}
Of note is the command \verb+\citet+, which produces citations
appropriate for use in inline text.  For example,
\begin{verbatim}
   \citet{hasselmo} investigated\dots
\end{verbatim}
produces
\begin{quote}
  Hasselmo, et al.\ (1995) investigated\dots
\end{quote}

\begin{center}
  \url{https://www.ctan.org/pkg/booktabs}
\end{center}


\subsection{Figures}
\lipsum[10] 
See Figure \ref{fig:fig1}. Here is how you add footnotes. \footnote{Sample of the first footnote.}
\lipsum[11] 

\begin{figure}
  \centering
  \fbox{\rule[-.5cm]{4cm}{4cm} \rule[-.5cm]{4cm}{0cm}}
  \caption{Sample figure caption.}
  \label{fig:fig1}
\end{figure}

\subsection{Tables}
\lipsum[12]
See awesome Table~\ref{tab:table}.

\begin{table}
 \caption{Sample table title}
  \centering
  \begin{tabular}{lll}
    \toprule
    \multicolumn{2}{c}{Part}                   \\
    \cmidrule(r){1-2}
    Name     & Description     & Size ($\mu$m) \\
    \midrule
    Dendrite & Input terminal  & $\sim$100     \\
    Axon     & Output terminal & $\sim$10      \\
    Soma     & Cell body       & up to $10^6$  \\
    \bottomrule
  \end{tabular}
  \label{tab:table}
\end{table}

\subsection{Lists}
\begin{itemize}
\item Lorem ipsum dolor sit amet
\item consectetur adipiscing elit. 
\item Aliquam dignissim blandit est, in dictum tortor gravida eget. In ac rutrum magna.
\end{itemize}


\bibliographystyle{unsrt}  
%\bibliography{references}  %%% Remove comment to use the external .bib file (using bibtex).
%%% and comment out the ``thebibliography'' section.


%%% Comment out this section when you \bibliography{references} is enabled.
\begin{thebibliography}{1}

\bibitem{kour2014real}
George Kour and Raid Saabne.
\newblock Real-time segmentation of on-line handwritten arabic script.
\newblock In {\em Frontiers in Handwriting Recognition (ICFHR), 2014 14th
  International Conference on}, pages 417--422. IEEE, 2014.

\bibitem{kour2014fast}
George Kour and Raid Saabne.
\newblock Fast classification of handwritten on-line arabic characters.
\newblock In {\em Soft Computing and Pattern Recognition (SoCPaR), 2014 6th
  International Conference of}, pages 312--318. IEEE, 2014.

\bibitem{hadash2018estimate}
Guy Hadash, Einat Kermany, Boaz Carmeli, Ofer Lavi, George Kour, and Alon
  Jacovi.
\newblock Estimate and replace: A novel approach to integrating deep neural
  networks with existing applications.
\newblock {\em arXiv preprint arXiv:1804.09028}, 2018.

\end{thebibliography}


\end{document}
